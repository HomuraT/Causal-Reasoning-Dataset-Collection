\section{因果推断 / Causal Inference}

\begin{frame}[label=cladder]{cladder: Causal Ladder Reasoning (1/2)}
\textbf{任务定义 / Task Definition}
\begin{itemize}
  \item 测试语言模型在 Pearl 因果阶梯三个层级上的推理能力:
  \[
    \text{Association} \rightarrow \text{Intervention} \rightarrow \text{Counterfactual}
  \]
  \item 每条样本由前提 $x$ 与因果询问 $q$ 组成,要求判断查询陈述是否成立。
\end{itemize}

\medskip
\textbf{输入 / 输出格式}
\begin{itemize}
  \item 输入:$x = \text{context},\ q = \text{causal query}$
  \item 输出标签集合:
  \[
    Y = \{\texttt{yes},\texttt{no}\} \quad \text{或} \quad Y=\{A,B\}
  \]
\end{itemize}

\medskip
\textbf{形式化定义}
\begin{itemize}
  \item 判定函数:
  \[
    f(x,q)=y,\quad y \in Y
  \]
  \item 每条样本带有整数阶梯标签 $\mathrm{rung} \in \{1,2,3\}$,大致对应相关性 (association)、干预 (intervention)、反事实 (counterfactual),并结合 \texttt{query\_type}(如 \texttt{marginal} / \texttt{ate} / \texttt{det-counterfactual} 等)做细粒度评估。
\end{itemize}
\end{frame}

\begin{frame}{cladder: Causal Ladder Reasoning (2/2)}
\begin{examplebox}[示例 / Example]
\small
\textbf{Prompt}\\
``Husband has a direct effect on wife and alarm clock \dots{} If we disregard the mediation effect through wife, would husband positively affect alarm clock?''

\medskip\hrule\medskip

\textbf{Label}\\
\texttt{yes}
\end{examplebox}

\medskip
\textbf{标签与阶梯}
\begin{itemize}
  \item 任务类型:相关性 / 干预 / 反事实三类因果问题。
  \item 选项数量:2 个 (yes/no 或 A/B)。
\end{itemize}
\end{frame}

\begin{frame}{cladder: Dataset Statistics (3/3)}
\textbf{整体规模与分布}
\begin{itemize}
  \item 样本总数:$N = 10112$。
  \item 每条样本带有阶梯标签 $\mathrm{rung} \in \{1,2,3\}$ 以及细粒度 \texttt{query\_type}。
\end{itemize}

\medskip
\begin{center}
  \includegraphics[width=0.9\linewidth]{images/CLADDER_statistic.png}
\end{center}
\end{frame}

\begin{frame}[label=copa]{copa: Everyday Causal QA (1/2)}
\textbf{任务定义}
\begin{itemize}
  \item 给定前提句 $p$ 和询问方向(\texttt{cause}/\texttt{effect}),在两个候选句中选出更合理的原因或结果:
  \[
    y \in \{A,B\}
  \]
\end{itemize}

\medskip
\textbf{输入 / 输出格式}
\begin{itemize}
  \item 输入:$p = \text{Premise},\ q \in \{\texttt{cause?},\texttt{effect?}\},\ \text{Options}=\{A,B\}$
  \item 输出:$Y = \{A,B\}$
\end{itemize}

\medskip
\textbf{形式化定义}
\begin{itemize}
  \item 决策函数:
  \[
    f(p,q,A,B) = y,\quad y = \arg\max_{c \in \{A,B\}} P(c \mid p,q)
  \]
\end{itemize}
\end{frame}

\begin{frame}{copa: Everyday Causal QA (2/2)}
\begin{examplebox}[示例]
\small
\textbf{Premise}\\
``The office was closed.''

\medskip\hrule\medskip

\textbf{Asks-for}\\
\texttt{cause}

\medskip\hrule\medskip

\textbf{Options}\\
\texttt{A) It was a holiday. \quad B) It was summer.}

\medskip\hrule\medskip

\textbf{Gold}\\
\texttt{A}
\end{examplebox}

\medskip
\textbf{选项与标签}
\begin{itemize}
  \item 选项数量:2 (A/B),输出集合 $Y=\{A,B\}$。
  \item 可分别统计 \texttt{cause} / \texttt{effect} 两种设问的准确率。
\end{itemize}
\end{frame}


