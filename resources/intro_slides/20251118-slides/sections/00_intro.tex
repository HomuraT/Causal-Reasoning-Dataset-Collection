\begin{frame}[plain]
  \titlepage
\end{frame}

\begin{frame}{Outline / 总目录}
  \footlinecenter{Causal-Reasoning-Dataset-Collection}
  \tableofcontents
\end{frame}

\begin{frame}{数据集总览 / Dataset Overview}
\small
\begin{tabular}{llll}
\toprule
\textbf{类别} & \textbf{数据集} & \textbf{任务形式} & \textbf{评价侧重点} \\
\midrule
\textbf{Causal Inference} & \hyperlink{cladder}{cladder} & 单分类(两类) & 区分相关/干预/反事实能力 \\
 & \hyperlink{copa}{copa} & 单分类(两类) & 语义常识下的因果方向判别 \\
\midrule
\textbf{Causal Discovery} & \hyperlink{corr2cause}{corr2cause} & 多分类(三类) & 相关与因果的区分能力 \\
 & \hyperlink{crab}{crab} & 排序 & 因果强度与成对一致性 \\
 & \hyperlink{crass}{crass} & 多分类(四类) & 反事实场景下的因果判断 \\
 & \hyperlink{e_care}{e\_care} & 单分类(两类) & 临床因果选择(原因/结果) \\
 & \hyperlink{pain}{pain} & 单分类(两类) & 临床因果真假判定 \\
\midrule
\textbf{Additional Causal Tasks} & \hyperlink{moca}{moca} & 单分类(两类) & 社会规范下的因果归因 \\
 & \hyperlink{tram}{tram} & 单分类(两类) & 现代语境的原因/结果选择 \\
\bottomrule
\end{tabular}

\vspace{0.5em}
\footnotesize 注:此处“单分类(两类)”与“多分类(三/四类)”均为单选题,区别仅在于标签类别数量的多少。
\end{frame}

\begin{frame}{常识相关性 / Commonsense Relevance}
\small
\begin{tabular}{lll}
\toprule
\textbf{数据集} & \textbf{任务形式} & \textbf{备注} \\
\midrule
\multicolumn{3}{l}{\textbf{近似纯符号 / 结构化推理(弱常识依赖)}} \\
\hyperlink{cladder}{cladder} & 单分类(两类) &
  Pearl 因果阶梯上的结构化因果推理 \\
\hyperlink{corr2cause}{corr2cause} & 多分类(三类) &
  相关 vs 因果的关系类型识别 \\
\midrule
\multicolumn{3}{l}{\textbf{显式依赖语义常识 / 领域背景}} \\
\hyperlink{copa}{copa} & 单分类(两类) &
  日常语境下的因果方向判别 \\
\hyperlink{crab}{crab} & 排序 &
  因果强度排序与成对一致性 \\
\hyperlink{crass}{crass} & 多分类(四类) &
  反事实场景下的常识性因果判断 \\
\hyperlink{e_care}{e\_care} & 单分类(两类) &
  临床场景下的因果选择(原因/结果) \\
\hyperlink{pain}{pain} & 单分类(两类) &
  临床因果陈述真伪判断 \\
\hyperlink{moca}{moca} & 单分类(两类) &
  道德与常识相关的因果归因 \\
\hyperlink{tram}{tram} & 单分类(两类) &
  现代语境下的原因/结果选择 \\
\bottomrule
\end{tabular}

\vspace{0.5em}
\footnotesize 上半部分数据集主要考察「给定结构/符号下的因果推理」,下半部分数据集则显式依赖语义常识、临床知识或社会规范。
\end{frame}


